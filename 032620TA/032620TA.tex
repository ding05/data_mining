\documentclass[10pt]{article}
\usepackage[utf8]{inputenc}
\usepackage[doublespacing]{setspace}
\usepackage{textcomp}
\usepackage{amsmath,amssymb,amsthm}
\usepackage{fancyhdr}
\usepackage{lastpage}
\usepackage[]{hyperref}
\usepackage[pdftex]{graphicx}
\usepackage{ctex}
\usepackage{booktabs}
\usepackage{subfigure}
\usepackage{titlesec}
\usepackage{listings}
\usepackage{enumerate}
\usepackage{bm}
\usepackage{float}
\usepackage{url}
\usepackage[english]{babel}
%\allowdisplaybreaks
\renewcommand{\contentsname}{\centerline{Contents}}
\pagestyle{fancy}
\author{D}
\def\name{D}
\lhead{Big Data}
\chead{}
\rhead{\name}
\cfoot{-\space\thepage\space-}
\newtheorem{exer}{\bm{$Exercise$}}
\newtheorem{prob}{\bm{$Problem$}}
\newtheorem{bonus}{\bm{$Bonus\;Problem$}}
\newcommand{\tabincell}[2]{\begin{tabular}{@{}#1@{}}#2\end{tabular}}
\CTEXoptions[today=old]

\begin{document}

\title{Assignment One}
\date{\today}
\maketitle
\thispagestyle{fancy}
\thispagestyle{fancy}

\begin{prob}
\end{prob}
\begin{enumerate}[1)]
\vspace{3mm}

\item
The observations for a response random variable $\pmb{y}$ are written as a vector, where
\begin{align*}
\pmb{y}=
  \begin{bmatrix}
    5\\
    9\\
    13
  \end{bmatrix}
,
\end{align*}
and the observations for the explanatory variable $\pmb{X}$ are written as a matrix, where
\begin{align*}
\pmb{X}=
  \begin{bmatrix}
    1 & 3\\
    1 & 4\\
    1 & 5
  \end{bmatrix}
.
\end{align*}
Referring to page 51 of the simple linear regression lecture notes\footnote{Reale, M. (2020). \textit{Lecture notes in big data}. Unpublished manuscript.}, straightforwardly, we get the ordinary least squares estimates of the coefficients, which are written as a matrix $\pmb{\hat{\beta}}$, where
\begin{align*}
\pmb{\hat{\beta}}&=(\pmb{X\textsuperscript{$\prime$}}\pmb{X})^{-1}\pmb{X\textsuperscript{$\prime$}}\pmb{y}.
\end{align*}
Based on the lecture notes, by hand we have two approaches to acquire $\pmb{\hat{\beta}}$.\\
The first approach is direct matrix calculation.\\
\begin{align*}
\pmb{\hat{\beta}}&=(\pmb{X\textsuperscript{$\prime$}}\pmb{X})^{-1}\pmb{X\textsuperscript{$\prime$}}\pmb{y}\\
&=(
  \begin{bmatrix}
    1 & 3\\
    1 & 4\\
    1 & 5
  \end{bmatrix}
'
  \begin{bmatrix}
    1 & 3\\
    1 & 4\\
    1 & 5
  \end{bmatrix}
)^{-1}
  \begin{bmatrix}
    1 & 3\\
    1 & 4\\
    1 & 5
  \end{bmatrix}
'
  \begin{bmatrix}
    5\\
    9\\
    13
  \end{bmatrix}
\\
&=
  \begin{bmatrix}
    3 & 12\\
    12 & 50
  \end{bmatrix}
^{-1}
  \begin{bmatrix}
    1 & 3\\
    1 & 4\\
    1 & 5
  \end{bmatrix}
'
  \begin{bmatrix}
    5\\
    9\\
    13
  \end{bmatrix}
\\
&=
  \begin{bmatrix}
    \frac{25}{3} & -2\\
    -2 & \frac{1}{2}
  \end{bmatrix}
  \begin{bmatrix}
    1 & 3\\
    1 & 4\\
    1 & 5
  \end{bmatrix}
'
  \begin{bmatrix}
    5\\
    9\\
    13
  \end{bmatrix}
\\
&=
  \begin{bmatrix}
    \frac{7}{3} & \frac{1}{3} & -\frac{5}{3}\\
    -\frac{1}{2} & 0 & \frac{1}{2}\\
  \end{bmatrix}
  \begin{bmatrix}
    5\\
    9\\
    13
  \end{bmatrix}
\\
&=
  \begin{bmatrix}
    -7\\
    4
  \end{bmatrix}
.
\end{align*}
The second approach is using the formulas on page 51.\\
Since $(\pmb{X\textsuperscript{$\prime$}}\pmb{X})^{-1}=\frac{1}{n\sum(x_i-\bar{x})^2}
  \begin{bmatrix}
    \sum{x_i^2} & -\sum{x_i}\\
    -\sum{x_i} & n
  \end{bmatrix}
$
and $\pmb{X\textsuperscript{$\prime$}}\pmb{y}=
  \begin{bmatrix}
    \sum{y_i}\\
    \sum{x_i y_i}
  \end{bmatrix}
$, we have
\begin{align*}
\pmb{\hat{\beta}}=\frac{1}{n\sum(x_i-\bar{x})^2}
  \begin{bmatrix}
    \sum{x_i^2} & -\sum{x_i}\\
    -\sum{x_i} & n
  \end{bmatrix}
  \begin{bmatrix}
    \sum{y_i}\\
    \sum{x_i y_i}
  \end{bmatrix}
.
\end{align*}
Using the given matrices, we get $n=3$, $\bar{x}=4$, $\sum(x_i-\bar{x})^2=2$, $\sum{x^2_i}=50$, $-\sum{x_i}=-12$, $\sum{y_i}=27$ and $\sum{x_i y_i}=116$. Therefore,
\begin{align*}
\pmb{\hat{\beta}}&=\frac{1}{3\times2}
  \begin{bmatrix}
    50 & -12\\
    -12 & 3
  \end{bmatrix}
  \begin{bmatrix}
    27\\
    116
  \end{bmatrix}
\\
&=\frac{1}{6}
  \begin{bmatrix}
    50\times27-12\times116\\
    -12\times27+3\times116
  \end{bmatrix}
\\
&=\frac{1}{6}
  \begin{bmatrix}
    -42\\
    24
  \end{bmatrix}
\\
&=
  \begin{bmatrix}
    -7\\
    4
  \end{bmatrix}
\end{align*}
The two approaches reach the same result. Hence, we get the required coefficients $\hat{\beta}_0=-7$ and $\hat{\beta}_1=4$.

\item
Referring to page 53, we get the estimated residuals $\pmb{\hat{\epsilon}}$, where
\begin{align*}
\pmb{\hat{\epsilon}}&=\pmb{y}-\pmb{X\hat{\beta}}\\
&=
  \begin{bmatrix}
    5\\
    9\\
    13
  \end{bmatrix}
-
  \begin{bmatrix}
    -7+4\times3\\
    -7+4\times4\\
    -7+4\times5
  \end{bmatrix}
\\
&=
  \begin{bmatrix}
    5\\
    9\\
    13
  \end{bmatrix}
-
  \begin{bmatrix}
    5\\
    9\\
    13
  \end{bmatrix}
\\
&=
  \begin{bmatrix}
    0\\
    0\\
    0
  \end{bmatrix}
\end{align*}
Hence, the estimates of the residuals $\pmb{\hat{\epsilon}}$ are $
  \begin{bmatrix}
    0 & 0 & 0
  \end{bmatrix}\textsuperscript{$\prime$}$.

\item
Please see ``032620RA.Rmd''.

\item
Please see ``032620RA.Rmd''.

\end{enumerate}
\vspace{3mm}

\begin{prob}
\end{prob}
\begin{enumerate}[1)]
\vspace{3mm}

\item
Given $\pmb{X}$, where
\begin{align*}
\pmb{X}=
  \begin{bmatrix}
    1 & 2\\
    1 & 2\\
    1 & 2
  \end{bmatrix}
,
\end{align*}
and $\pmb{y}$ unchanged, we repeat all the approaches in the previous problem.\\
First, using matrix calculation by hand,
\begin{align*}
\pmb{\hat{\beta}}&=(\pmb{X\textsuperscript{$\prime$}}\pmb{X})^{-1}\pmb{X\textsuperscript{$\prime$}}\pmb{y}\\
&=(
  \begin{bmatrix}
    1 & 2\\
    1 & 2\\
    1 & 2
  \end{bmatrix}
'
  \begin{bmatrix}
    1 & 2\\
    1 & 2\\
    1 & 2
  \end{bmatrix}
)^{-1}
  \begin{bmatrix}
    1 & 2\\
    1 & 2\\
    1 & 2
  \end{bmatrix}
'
  \begin{bmatrix}
    5\\
    9\\
    13
  \end{bmatrix}
\\
&=
  \begin{bmatrix}
    3 & 6\\
    6 & 12
  \end{bmatrix}
^{-1}
  \begin{bmatrix}
    1 & 2\\
    1 & 2\\
    1 & 2
  \end{bmatrix}
'
  \begin{bmatrix}
    5\\
    9\\
    13
  \end{bmatrix}
.
\end{align*}
Since $
  \begin{vmatrix}
    3 & 6\\
    6 & 12
  \end{vmatrix}
=0$,$
  \begin{bmatrix}
    3 & 6\\
    6 & 12
  \end{bmatrix}
$ is not invertible. We can not proceed the calculation and thus can not acquire $\pmb{\hat{\beta}}$.\\
Second, using the formulas in the lecture notes, where $\pmb{\hat{\beta}}=(\pmb{X\textsuperscript{$\prime$}}\pmb{X})^{-1}\pmb{X\textsuperscript{$\prime$}}\pmb{y}=\frac{1}{n\sum(x_i-\bar{x})^2}
  \begin{bmatrix}
    \sum{x_i^2} & -\sum{x_i}\\
    -\sum{x_i} & n
  \end{bmatrix}
  \begin{bmatrix}
    \sum{y_i}\\
    \sum{x_i y_i}
  \end{bmatrix}$, since $\frac{1}{n\sum(x_i-\bar{x})^2}=\frac{1}{3\sum(0)^2}=\frac{1}{0}$ is undefined, we can not acquire $\pmb{\hat{\beta}}$.\\
Third, using matrix calculation in R, we get an non-conformable arguments error.\\
Fourth, using the function \textbf{lm} in R, we get the following outputs,\\
\begin{table}[H]
\centering
\begin{tabular}{rr}
\multicolumn{2}{l}{Coefficients:} \\
(Intercept)          & X          \\
9                    & NA
\end{tabular}
\end{table}
showing $\hat{\beta}_0=9$ and $\hat{\beta}_1$ is not applicable.\\
In all, the complete coefficient estimate matrix can not be acquired.

\item
A statistical explanation.\\
We recall the matrix calculation, where
\begin{align*}
\pmb{\hat{\beta}}&=(\pmb{X\textsuperscript{$\prime$}}\pmb{X})^{-1}\pmb{X\textsuperscript{$\prime$}}\pmb{y}\\
&=(
  \begin{bmatrix}
    1 & 2\\
    1 & 2\\
    1 & 2
  \end{bmatrix}
'
  \begin{bmatrix}
    1 & 2\\
    1 & 2\\
    1 & 2
  \end{bmatrix}
)^{-1}
  \begin{bmatrix}
    1 & 2\\
    1 & 2\\
    1 & 2
  \end{bmatrix}
'
  \begin{bmatrix}
    5\\
    9\\
    13
  \end{bmatrix}
\\
&=
  \begin{bmatrix}
    3 & 6\\
    6 & 12
  \end{bmatrix}
^{-1}
  \begin{bmatrix}
    1 & 2\\
    1 & 2\\
    1 & 2
  \end{bmatrix}
'
  \begin{bmatrix}
    5\\
    9\\
    13
  \end{bmatrix}
.
\end{align*}
Since the determinant of $
  \begin{bmatrix}
    3 & 6\\
    6 & 12
  \end{bmatrix}
$ is $
  \begin{vmatrix}
    3 & 6\\
    6 & 12
  \end{vmatrix}
$, where $
  \begin{vmatrix}
    3 & 6\\
    6 & 12
  \end{vmatrix}
=0$,
$
  \begin{bmatrix}
    3 & 6\\
    6 & 12
  \end{bmatrix}
$ is singular (not full rank) and thus not invertible. Hence, $\pmb{\hat{\beta}}$ can not be acquired.\\
A geometric explanation.\\
\begin{figure}[H]
  \centering
  \includegraphics[scale=0.5]{p22a.png}
  \caption{Geometry of OLS}
\end{figure}
Let $\pmb{H}$ be a hat matrix, where $\pmb{\hat{y}}=\pmb{H}\pmb{y}$, so we have
\begin{align*}
\pmb{H}\pmb{y}=\hat{\beta}_0\pmb{x_0}+\hat{\beta}_1\pmb{x_1}.
\end{align*}
Geometrically, $\pmb{\hat{y}}$ is the orthogonal projection of $\pmb{y}$; $cos(\theta)$ implies the correlation between $\pmb{y}$ and $\pmb{X}$.\\
In this case where $cos(\theta)=0$, $\pmb{y}$ is perpendicular to the $x_0x_1$ plane, the projection $\pmb{\hat{y}}$ is $\pmb{0}$ and the correlation is zero, i.e. unexplainability.\\
Use the graph. Lines in gray show the general form (full rank); lines in black are for this case (not full rank).
\end{enumerate}

\end{document}
